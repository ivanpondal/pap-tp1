\section{Ejercicio 2}

Peso asignado: 9.
\subsection{Problema}
El enunciado explica un problema en el cual se tiene una cierta cantidad de
`amigas' de Lisa $N$ y para cada una de ellas se sabe la diversión que cada par
de amigas aporta a una fiesta. Luego se pide hallar la máxima diversión que se
puede obtener separando a sus amigas en `fiestas' donde cada fiesta le aporta la
diversión que suman las amigas que están en ella. \\ Yendo al algoritmo, se
tiene como entrada un valor entero positivo $N$ y una
matriz $D$ de $N \times N$ simétrica con valores enteros y diagonal 0.

Esto tiene sentido ya que una amiga no aporta diversión consigo misma (por eso
la diagonal en 0) y la diversión que aporta una amiga $j$ con otra amiga $t$ es
la misma diversión que aporta $t$ con $j$ (por eso es simétrica).

Tal como relata el problema, $N$ es la cantidad de amigas de Lisa y la matriz
$D$ dice para cada $t,j \in [1..n]$ la diversión que aportan las amigas $t$
y $j$ si se encontraran en la misma fiesta.

Dado el conjunto $A = \{1,...,N\}$ (que se usarán para representar a las amigas
de Lisa) y una partición $B = \{A_1,...,A_k\}$ de A, definimos $B_D$ la `suma de
una partición sobre $D$' como:

\begin{equation*}
    B_D = \sum_{i=1}^{k}{\sum_{j \in A_i}^{}{\sum_{t \in A_i : t < j}^{}{D_{t,j}}}}
\end{equation*}

Es decir, $B_D$ es la suma de $D_{t,j}$ para todo $t$ y $j$ que pertenecen al
mismo subconjunto en la partición $B$, la condición $t < j$ asegura no
considerar $D_{t,j}$ y $D_{j,t}$ ya que la matriz es simétrica y solo
interesa considerar cuánto le aporta a la `suma' el par una sola vez.

A su vez, $B_D$ representa cuánta diversión se tendría si se separan las
amigas de Lisa en fiestas diferentes mediante la partición $B$.

Habiendo definido eso, se quiere hallar el máximo $B_D$ para cualquier partición
$B$ de $A$, mejor dicho, la máxima `suma de una partición sobre D' de $A$
posible, lo cual representaría la forma de separar a las amigas de Lisa en
fiestas para que la suma de la diversión que aportan sea máxima.

\subsection{Algoritmo e intuición}

\subsubsection*{Pseudocódigo}

\begin{algorithm}[H]
    \caption{MáximaSumaPartición}
    \Input{Entero positivo $N$ y Matriz $D$ simétrica de enteros con diagonal 0}
    $int$  $\mathit{dp[1<<N]} \gets [-1,...,-1]$ \;
    $\mathit{dp[0]} \gets 0$ \;
    \Return{\emph{MáximaSumaParticiónParaSubconjunto}$(N,D,dp,(1<<N)-1)$}
\end{algorithm}

\begin{algorithm}[H]
    \caption{MáximaSumaParticiónParaSubconjunto}
    \Input{Entero positivo $N$, Matriz $D$ simétrica de enteros con diagonal 0, Arreglo $dp$ de tamaño $2^N$ y Mascara de bits $mask$}
    \eIf{$\mathit{dp[mask]}$ != $-1$} {
        \Return{$\mathit{dp[mask]}$}
    }{
    int $\mathit{res} \gets 0$ \;
    \For {int $i \gets 0$ ; $i < N$; $i \gets i + 1$} {
        \For {int $j \gets 0$ ; $j < i$; $j \gets j + 1$} {
            \If {($mask$ \& (($1 << i$) $\|$ ($1 << j$))) $==$ (($1 << i$) $\|$ ($1 << j$))} {
                $res$ $+=$ $D_{i,j}$
            }
        }
    }
    $res \gets max(res, 0)$ \;
    \For {int $s1 \gets mask \& (mask-1)$; $s1$ != $0$; $s1 \gets mask \& (s1-1)$} {
        int $s2 \gets mask \& ($ $NOT$ $ s1)$ \;
        int $c1$ = \emph{MáximaSumaParticiónParaSubconjunto}$(N, D, dp, s1)$ \;
        int $c2$ = \emph{MáximaSumaParticiónParaSubconjunto}$(N, D, dp, s2)$ \;
        $res \gets max(res, c1+c2)$ \;
    }
    $dp[mask] \gets res$ \;
    \Return{$res$}
    }
\end{algorithm}
\bigskip

\subsubsection*{Estrategia}

El algoritmo usa la estrategia de Programación Dinámica para iterar
conjuntos que son representados con máscaras de bits. Comienza llamando a la
función recursiva \emph{MáximaSumaParticiónParaSubconjunto} con la máscara que
representa al conjunto entero y se pide la máxima diversión que pueden
aportar.

Luego la función recursiva se fija para todos sus particiones
posibles, de qué forma puede dividir el conjunto tal que la diversión que
aporten los subconjuntos de la partición sea máxima. Nuevamente, esto lo se hace
representando los subconjuntos con máscaras de bits y recorriendo los
subconjuntos de cada conjunto de la forma que fuer explicada en clase.

\subsection{Correctitud}

\subsubsection*{MáximaSumaPartición}
Tal como fue visto en ejemplos en clase, inicializamos un arreglo de soluciones
para todos los subconjuntos posibles de `amigas de Lisa' que fueron llamados
$dp$, donde $dp[mask]$ al final del algoritmo debería haber calculado para el
subconjunto de amigas que representa `mask', la máxima suma posible de diversión
que pueden aportar esas amigas repartiéndolas (particionándolas) en fiestas.

Inicialmente se quiere calcular la máxima diversión que puede obtenerse mediante
todas las amigas de Lisa, lo cual se representa mediante la máscara de bits
`11...1' (donde la cantidad de 1s es N), lo cual se escribe como $(1<<N)-1$.

Dada esta preparación se manda $dp$ y la máscara inicial al algoritmo
\emph{MáximaSumaParticiónParaSubconjunto} que calcula la máxima diversión para
todas las amigas de Lisa.

\subsubsection*{MáximaSumaParticiónParaSubconjunto}

En este algoritmo, dadas las soluciones recursivas ya calculadas en el arreglo
de enteros $dp$, toma una máscara de bits $mask$ y calcula la máxima diversión
que pueden aportar las amigas de Lisa que representa $mask$.

En primer lugar se fija si la solución ya fue calculada previamente en $dp$
y si lo fue la devolvemos. En caso contrario, se calculan tres cosas:

\begin{itemize}
    \item [$res \gets 0$] El caso en que todas las amigas van a fiestas distintas (la diversión sería 0).
    \item [For 1] Cuando todas las amigas de el subconjunto actual van a la misma fiesta (lo que se calcula con el primer ciclo For). Acá lo que tiene que hacerse es simplemente recorrer cada par $i,j$ con $i,j \in [1..N]$ (pero no el par $j,i$, lo cual ya fue explicado al comienzo) y sumar la diversión que aportan si es que las amigas $i$ y $j$ de Lisa están en el subconjunto actual. Ver que están en el subconjunto actual (representado por $mask$) puede verse exactamente de la misma forma que fue realizada en clase: \\
    ($mask$ \& (($1 << i$) $\|$ ($1 << j$))) $==$ (($1 << i$) $\|$ ($1 << j$)).
    \item [For 2] Se ve para cada subconjunto propio del conjunto actual cuánta diversión puedo obtener si se separa el conjunto actual en dos: el subconjunto propio y su complemento (que obtenemos con $s2 \gets mask \& ($ $NOT$ $ s1)$). Con esto hacemos dos llamados recursivos a esta misma función y comparamos qué forma de partir el conjunto actual puede terminar aportando más diversión. Es importante ver que de esta manera se recorren todas las particiones posibles del conjunto actual ya que si bien en este ciclo For separamos el conjunto en dos de todas las formas posibles, cuando hago las llamadas recursivas, estos subconjuntos serán también particionados de todas las formas posibles. Como fue visto en clase, recorrer los subconjuntos de un conjunto se puede hacer con: \\
    int $s1 \gets mask$; $s1$ != $0$; $s1 \gets mask \& (s1-1)$ \\
    Pero en este código se quiere recorrer los subconjuntos propios, por lo cual se modifica a: \\
    int $s1 \gets mask \& (mask-1)$; $s1$ != $0$; $s1 \gets mask \& (s1-1)$ \\
    Que simplemente comienza con el primer subconjunto diferente al conjunto actual.
\end{itemize}

De esas tres opciones se elige como solución la que más diversión aporta.

Finalmente se guarda la solución para el subconjunto actual en el arreglo $dp$ y se la devuelve como respuesta.

\subsection{Complejidad}

Se recorren todas las $2^N$ instancias representadas con las máscaras de bits
mediante llamados recursivos, pero no más que eso porque se guardan las
respuestas ya calculadas con Programación Dinámica mediante un arreglo.

Por cada instancia se tienen algunas asignaciones y dos ciclos For.

El primer ciclo itera $O(n^2)$ combinaciones de indices y adentro realiza
comparaciones y asignaciones por lo cual su complejidad es de $O(n^2)$.

El segundo ciclo itera cada subconjunto del conjunto actual y adentro realiza
(además de los llamados recursivos, que eso fue considerado aparte al principio
cuando dijimos que iteramos los $2^N$ subconjuntos posibles), solamente
asignaciones y otras operaciones básicas.

Como se vio en clase, si para cada instancia de las $2^N$ que existen se iteran
sus subconjuntos, da una complejidad de $O(3^N)$.\\ Además, por cada una de
las instancias se realiza $O(n^2)$ operaciones mediante el primer For, lo cual
conlleva una complejidad de $O(2^n * n^2) = O(3^n)$.

En total se tiene una complejidad de $O(3^n) + O(3^n) = O(3^n)$, tal como pide
el enunciado.

\subsection{Casos de prueba}

En primera instancia se comprobó que el algoritmo devuelva la salida correcta para los casos provistos por la cátedra. \\
Luego fueron generadas instancias propias bajo las condiciones del enunciado, es decir, matrices cuadradas y simétricas con diagonal 0 de tamaño menor o igual a 18. \\
Mediante estos, se comprobó no solo la correctitud misma del algoritmo sino también casos borde
(por ejemplo, que todas las amigas de Lisa se lleven mal entre sí)
y que no tarde demasiado en encontrar una respuesta para instancias grandes.


\subsubsection*{Caso 1}

\textbf{Motivo}: Se comprobó que el algoritmo funcione correctamente en una instancia un poco mas grande que las provistas por la cátedra. \\

\textbf{Entrada}:

\begin{flushleft}
$\qquad N = 5$\\[10pt]
$\qquad D = \left( \begin{smallmatrix}
0 & 2 & 2 & -1 & 2 \\
2 & 0 & -10 & 1 & 1 \\
2 & -10 & 0 & -2 & 2 \\
-1 & 1 & -2 & 0 & 3 \\
2 & 1 & 2 & 3 & 0 \\
\end{smallmatrix} \right)$
\end{flushleft}

\textbf{Resultado}: El algoritmo dio como resultado 8 y lo hizo de manera inmediata. Esto tiene sentido ya 
que es una instancia chica y se llega a esa solución tomando las amigas $\{ 0, 1, 3, 4 \}$ en una fiesta y $\{ 2 \}$ en otra fiesta.

\subsubsection*{Caso 2}

\textbf{Motivo}: Se comprobó el funcionamiento del algoritmo para una instancia un poco más grande donde tenemos 3 clanes de amigas de Lisa bien definidos. 
Cada clan es de 4 amigas que se llevan bien entre sí pero con nadie más de otros clanes. \\

\textbf{Entrada}:

\begin{flushleft}
$\qquad N = 12$\\[10pt]
$\qquad D = \left( \begin{smallmatrix}
0 & 1 & 1 & 1 & -1 & -1 & -1 & -1 & -1 & -1 & -1 & -1 \\
1 & 0 & 1 & 1 & -1 & -1 & -1 & -1 & -1 & -1 & -1 & -1 \\
1 & 1 & 0 & 1 & -1 & -1 & -1 & -1 & -1 & -1 & -1 & -1 \\
1 & 1 & 1 & 0 & -1 & -1 & -1 & -1 & -1 & -1 & -1 & -1 \\
-1 & -1 & -1 & -1 & 0 & 1 & 1 & 1 & -1 & -1 & -1 & -1 \\
-1 & -1 & -1 & -1 & 1 & 0 & 1 & 1 & -1 & -1 & -1 & -1 \\
-1 & -1 & -1 & -1 & 1 & 1 & 0 & 1 & -1 & -1 & -1 & -1 \\
-1 & -1 & -1 & -1 & 1 & 1 & 1 & 0 & -1 & -1 & -1 & -1 \\
-1 & -1 & -1 & -1 & -1 & -1 & -1 & -1 & 0 & 1 & 1 & 1 \\
-1 & -1 & -1 & -1 & -1 & -1 & -1 & -1 & 1 & 0 & 1 & 1 \\
-1 & -1 & -1 & -1 & -1 & -1 & -1 & -1 & 1 & 1 & 0 & 1 \\
-1 & -1 & -1 & -1 & -1 & -1 & -1 & -1 & 1 & 1 & 1 & 0 \\
\end{smallmatrix} \right)$
\end{flushleft}

\textbf{Resultado}: El algoritmo dio como resultado 18 y lo hizo de manera casi inmediata (no lento, pero no
 tan instantáneo como el caso anterior). Como podía predecirse, el algoritmo da el resultado de agrupar 
 cada uno de los 3 clanes de los que se habló previamente. Cada amiga en un clan aporta 1 punto de diversión 
 con las demás, por lo tanto un clan de 4 amigas suma en total 6 puntos de diversión, es decir el número 
 combinatorio (4,2) que son las formas de tomar 2 amigas de un clan de 4. Si se suma la diversión de cada fiesta
 de cada clan nos da $3 * 6 = 18$ unidades de diversión.

\subsubsection*{Caso 3}

\textbf{Motivo}: Se quiere comprobar cuánto tiempo tarda el algoritmo para una instancia de tamaño máximo y ver el resultado para el caso en que ninguna amiga de Lisa se lleva bien con las demás.

\textbf{Entrada}:

\begin{flushleft}
$\qquad N = 18$\\[10pt]
$\qquad D = \left( \begin{smallmatrix}
0 & -1 & -1 & -1 & -1 & -1 & -1 & -1 & -1 & -1 & -1 & -1 & -1 & -1 & -1 & -1 & -1 & -1 \\
-1 & 0 & -1 & -1 & -1 & -1 & -1 & -1 & -1 & -1 & -1 & -1 & -1 & -1 & -1 & -1 & -1 & -1 \\
-1 & -1 & 0 & -1 & -1 & -1 & -1 & -1 & -1 & -1 & -1 & -1 & -1 & -1 & -1 & -1 & -1 & -1 \\
-1 & -1 & -1 & 0 & -1 & -1 & -1 & -1 & -1 & -1 & -1 & -1 & -1 & -1 & -1 & -1 & -1 & -1 \\
-1 & -1 & -1 & -1 & 0 & -1 & -1 & -1 & -1 & -1 & -1 & -1 & -1 & -1 & -1 & -1 & -1 & -1 \\
-1 & -1 & -1 & -1 & -1 & 0 & -1 & -1 & -1 & -1 & -1 & -1 & -1 & -1 & -1 & -1 & -1 & -1 \\
-1 & -1 & -1 & -1 & -1 & -1 & 0 & -1 & -1 & -1 & -1 & -1 & -1 & -1 & -1 & -1 & -1 & -1 \\
-1 & -1 & -1 & -1 & -1 & -1 & -1 & 0 & -1 & -1 & -1 & -1 & -1 & -1 & -1 & -1 & -1 & -1 \\
-1 & -1 & -1 & -1 & -1 & -1 & -1 & -1 & 0 & -1 & -1 & -1 & -1 & -1 & -1 & -1 & -1 & -1 \\
-1 & -1 & -1 & -1 & -1 & -1 & -1 & -1 & -1 & 0 & -1 & -1 & -1 & -1 & -1 & -1 & -1 & -1 \\
-1 & -1 & -1 & -1 & -1 & -1 & -1 & -1 & -1 & -1 & 0 & -1 & -1 & -1 & -1 & -1 & -1 & -1 \\
-1 & -1 & -1 & -1 & -1 & -1 & -1 & -1 & -1 & -1 & -1 & 0 & -1 & -1 & -1 & -1 & -1 & -1 \\
-1 & -1 & -1 & -1 & -1 & -1 & -1 & -1 & -1 & -1 & -1 & -1 & 0 & -1 & -1 & -1 & -1 & -1 \\
-1 & -1 & -1 & -1 & -1 & -1 & -1 & -1 & -1 & -1 & -1 & -1 & -1 & 0 & -1 & -1 & -1 & -1 \\
-1 & -1 & -1 & -1 & -1 & -1 & -1 & -1 & -1 & -1 & -1 & -1 & -1 & -1 & 0 & -1 & -1 & -1 \\
-1 & -1 & -1 & -1 & -1 & -1 & -1 & -1 & -1 & -1 & -1 & -1 & -1 & -1 & -1 & 0 & -1 & -1 \\
-1 & -1 & -1 & -1 & -1 & -1 & -1 & -1 & -1 & -1 & -1 & -1 & -1 & -1 & -1 & -1 & 0 & -1 \\
-1 & -1 & -1 & -1 & -1 & -1 & -1 & -1 & -1 & -1 & -1 & -1 & -1 & -1 & -1 & -1 & -1 & 0 \\
\end{smallmatrix} \right)$
\end{flushleft}

\textbf{Resultado}: El algoritmo dió como resultado 0, lo cual era de esperarse ya que ninguna amiga de Lisa 
se llevaba bien con las demás y por lo tanto se deben separar en una fiesta por amiga. En cuanto al tiempo que 
tardó en dar esta respuesta fue de un par de segundos, lo cual indica que el tiempo aumenta bastante a medida 
que crecen las instancias pero aún así lo resuelve en tiempo razonable para las cotas dadas por la cátedra.
