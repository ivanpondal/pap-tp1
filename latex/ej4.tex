\section{Ejercicio 4}

\subsection{Introducción}

Para este ejercicio se pedía dado un arreglo de $N$ matrices en
$\mathbb{Z}_{10007}^{3 \times 3}$ decidir si existía un subarreglo de longitud
$L$ tal que su producto fuera igual a $M \in \mathbb{Z}_{10007}^{3 \times 3}$.
Además, el algoritmo desarrollado debía tener una complejidad temporal $\ord(N
\times \log N)$.

\subsection{Solución propuesta}

La solución desarrollada hace uso de la técnica de \emph{Divide \& Conquer}.
Esto se debe a que el problema tiene la característica de poder ser dividido en
subproblemas más pequeños que unidos resuelven lo pedido.

Si existe el subarreglo cuyo producto corresponde a $M$ es bajo alguna de
las siguientes posibilidades:

\begin{itemize}
	\item El subarreglo existe en $\left[ 0,\frac{N}{2} \right)$.
	\item El subarreglo existe en $\left[\frac{N}{2}, N \right)$.
	\item El subarreglo está atravesando ambas mitades.
\end{itemize}

En caso de no cumplirse ninguna de estas opciones el subarreglo pedido no
existe.

De esta forma se puede ver entonces cómo partiendo el arreglo en dos
mitades el ejercicio se puede resolver llamando el algoritmo de forma recursiva
sobre cada una y estudiando el caso donde atraviesa a ambas.

A continuación se presenta el pseudocódigo que resuelve el problema:

\begin{algorithm}[H]
	\caption{Producto subarreglo de matrices}
	\Input{Enteros positivos $N$ y $L$, índices $i$ y $j$, una matriz $M$ y un
	arreglo de matrices de longitud $N$ pertenecientes a $\mathbb{Z}_{10007}^{3
	\times 3}$.}
	\Output{Devuelve \texttt{true} en caso de existir un subarreglo cuyo producto
	sea igual a $M$, \texttt{false} caso contrario.}
	\eIf{$N == 1$} {
		\eIf{$L == 1$ y el único elemento del arreglo es igual a $M$} {
			\Return{\texttt{true}} \;
		}
		{
			guardar el elemento en la posición $i$ del arreglo de productos \;
		}
	}
	{
		\eIf{$L \leq N$} {
			$mitad$ $\gets$ $\frac{N}{2}$ \;
			$derechaResuelve$ $\gets$ llamada recursiva con $N = N - mitad$, $i =
			i + mitad$ \;
			$izquierdaResuelve$ $\gets$ llamada recursiva con $N = mitad$, $j = i
			+ mitad$ \;
			\eIf{$derechaResuelve$ ó $izquierdaResuelve$} {
				\Return{\texttt{true}} \;
			}
			{
				\eIf{existe producto de longitud $L$ con resultado $M$ que
				atraviese las mitades} {
					\Return{\texttt{true}} \;
				}
				{
					calcular y guardar el producto de matrices entre $i$ y $j$ \;
				}
			}
		}
		{
			calcular y guardar el producto de matrices entre $i$ y $j$ \;
		}
	}

	\Return{\texttt{false}} \;
\end{algorithm}


