\section{Ejercicio 4}

\subsection{Introducción}

Para este ejercicio se pedía dado un arreglo de $N$ matrices en
$\mathbb{Z}_{10007}^{3 \times 3}$ decidir si existía un subarreglo de longitud
$L$ tal que su producto fuera igual a $M \in \mathbb{Z}_{10007}^{3 \times 3}$.
Además, el algoritmo desarrollado debía tener una complejidad temporal $\ord(N
\times \log N)$.

\subsection{Solución propuesta}\label{ej4:sol}

La solución desarrollada hace uso de la técnica de \emph{Divide \& Conquer}.
Esto se debe a que el problema tiene la característica de poder ser dividido en
subproblemas más pequeños que unidos resuelven lo pedido.

Si existe el subarreglo cuyo producto corresponde a $M$ es bajo alguna de
las siguientes posibilidades:

\begin{itemize}
	\item El subarreglo existe en $\left[ 0,\frac{N}{2} \right)$.
	\item El subarreglo existe en $\left[\frac{N}{2}, N \right)$.
	\item El subarreglo está atravesando ambas mitades.
\end{itemize}

En caso de no cumplirse ninguna de estas opciones el subarreglo pedido no
existe.

De esta forma se puede ver entonces cómo partiendo el arreglo en dos
mitades el ejercicio se puede resolver llamando el algoritmo de forma recursiva
sobre cada una y estudiando el caso donde atraviesa a ambas.

El escenario que requiere mayor atención es el del subarreglo atravesando ambas mitades.
Para ello, es necesario probar el producto de todo subarreglo de longitud $L$
que se encuentre atravesando ambas mitades.

Una implementación básica simplemente intentaría multiplicar desde la posición
más a la izquierda tal que el subarreglo resultante de tamaño $L$ atravesara la
mitad. Luego, si el producto fuera distinto a $M$, se iría corriendo de a un
lugar a la derecha repitiendo el proceso hasta que ya no fuera posible o
encontrase el subarreglo buscado. Esta operación sería cuadrática en el tamaño
de $L$, que al estar acotado por $N$ resulta en una complejidad temporal
$\ord(N^{2})$. Como esta cota supera la requerida por el ejercicio fue necesario
optimizar esta operación.

Dado que el producto de matrices es asociativo, teniendo un subarreglo de tamaño
$L$ con elementos $M_k \dots M_{k + L - 1}$ su producto se puede calcular
mediante:

\begin{gather*}
	\left(M_k \times M_{k + 1} \times \dots \times M_m\right) \times
		\left(M_{m + 1} \times M_{m + 2} \times \dots \times M_{k + L - 1}\right) \\
		\left(\forall m\right) k \leq m \leq k + L - 1
\end{gather*}

Entonces, si definimos $m = \frac{N}{2}$ el producto de todos los subarreglos de
tamaño $L$ que atraviesan ambas mitades se puede calcular como el de las
matrices de la mitad izquierda por el de las de la derecha.

Para reducir el número de operaciones lo que se hace es para cada mitad ya tener
calculados los productos de matrices desde la mitad hacia el extremo del
arreglo. Cuando se realiza la llamada recursiva a la función, además de pasar el
intervalo del arreglo sobre el cual operar, se indica mediante una bandera si se
trata de la mitad derecha o la izquierda. Si se trata de la mitad derecha los
productos se calculan desde el primer elemento hacia el último, caso
contrario, del último a el primero.

Esta operación tiene un costo $\ord(N)$ ya que para cada posición del arreglo se
calcula y almacena el producto entre la matriz en ese punto y el producto hasta
la posición anterior.

\begin{figure}[H]
	\centering
	\includegraphics{imagenes/subarray_product_across.pdf}
	\caption{Ejemplo de subarreglo atravesando ambas mitades con $L = 4$ y $N = 8$.}
	\label{ej4:fig:subarray}
\end{figure}

En la Figura \ref{ej4:fig:subarray} se tienen en color claro los elementos del
subarreglo pertenecientes a la mitad izquierda y en color oscuro los de la
derecha. El algoritmo calcula este producto como \texttt{producto[1]} $\times$
\texttt{producto[4]}, donde \texttt{producto[1]} fue calculado por la mitad
izquierda y \texttt{producto[4]} por la derecha:

\begin{align*}
	\texttt{producto[1]} &= M_1 \times \texttt{producto[2]} =
	M_1 \times \left(M_2 \times \texttt{producto[3]}\right) =
	M_1 \times \left(M_2 \times M_3\right) \\
	\texttt{producto[4]} &= M_4
\end{align*}

Es así como la operación responsable de analizar si existe el subarreglo
atravesando ambas mitades pasa a tener un costo $\ord(N)$ ya que únicamente
realiza $\ord(L)$ productos de factores que ya posee generados.

A continuación se presenta el pseudocódigo del algoritmo que resuelve el problema:

\begin{algorithm}[H]
	\caption{Producto subarreglo de matrices}
	\Input{Enteros positivos $N$ y $L$, índices $i$ y $j$, bandera
		\emph{esMitadDerecha} indicando en qué mitad está corriendo la función,
		una matriz $M$ y un arreglo de matrices de longitud $N$ pertenecientes a
		$\mathbb{Z}_{10007}^{3 \times 3}$.}
	\Output{Devuelve \texttt{true} en caso de existir un subarreglo de longitud
		$L$ cuyo producto sea igual a $M$, \texttt{false} caso contrario.}
	\eIf{$N == 1$} {
		\If{$L == 1$ y el único elemento del arreglo es igual a $M$} {
			\Return{\texttt{true}} \;
		}
	}
	{
		\eIf{$L \leq N$} {
			$mitad$ $\gets$ $\frac{N}{2}$ \;
			$derechaResuelve$ $\gets$ llamada recursiva con $N = N - mitad$, $i =
			i + mitad$ \;
			$izquierdaResuelve$ $\gets$ llamada recursiva con $N = mitad$, $j = i
			+ mitad$ \;
			\eIf{$derechaResuelve$ ó $izquierdaResuelve$} {
				\Return{\texttt{true}} \;
			}
			{
				\eIf{el producto existe atravesando ambas mitades} {
					\Return{\texttt{true}} \;
				}
				{
					\eIf{\emph{esMitadDerecha}} {
						calcular y guardar el producto de matrices de $i$ a $j$ \;
					}
					{
						calcular y guardar el producto de matrices de $j$ a $i$ \;
					}
				}
			}
		}
		{
			\eIf{\emph{esMitadDerecha}} {
				calcular y guardar el producto de matrices de $i$ a $j$ \;
			}
			{
				calcular y guardar el producto de matrices de $j$ a $i$ \;
			}
		}
	}

	\Return{\texttt{false}} \;
\end{algorithm}


