\section{Ejercicio 3}

Peso asignado: 9.

\subsection{Problema}
Dado un arreglo de números enteros de entrada, se pide encontrar la máxima suma posible de elementos correspondientes a subarreglos contiguos de dicha entrada. Se considera que el arreglo vacío tiene suma nula y que cumple ser un subarreglo contiguo de todo arreglo. Consideraremos que un \textit{subarreglo} siempre es contiguo. Además, nos referirimos a la suma de los elementos de un subarreglo como \textit{la suma del subarreglo}.

\subsection{Algoritmo e intuición}

\begin{algorithm}[H]
	\caption{Máxima suma de subarreglos}
	\Input{Arreglo $A$ de números enteros}
    % \Output{Bla}
    $\mathit{acumulador, mayorAlcanzado} \gets 0$ \;
    \For{$x$ en $A$} {
        $\mathit{acumulador} \gets$ el máximo entre $\mathit{acumulador}+x$ y $0$ \;
        $\mathit{mayorAlcanzado} \gets$ el máximo entre $\mathit{mayorAlcanzado}$ y $\mathit{acumulador}$
    }
    \Return{$\mathit{mayorAlcanzado}$}
\end{algorithm}

\bigskip

Nuestro algoritmo itera los elementos del arreglo recorriéndolo de principio a fin una sola vez. En cada iteración, suma en un acumulador inicializado en cero el elemento correspondiente, siempre que al sumarlo la acumulación resulte positiva; de lo contrario, vuelve a poner el acumulador en cero. De esta forma, luego de la iteración $k$, \textit{acumulador} tiene el valor de la suma o bien de un subarreglo cuyo último elemento es el $k$\textit{-ésimo} del arreglo, o bien la del subarreglo vacío. Además, el algoritmo tiene en \textit{mayorAlcanzado} luego de cada iteración un registro del máximo valor de acumulación conseguido hasta ese momento.

La intuición detrás del algoritmo es que, mientras un subarreglo en particular sume una cantidad positiva, tiene sentido seguir considerándolo como el prefijo de un subarreglo candidato a realizar la suma óptima, ya que esa suma puede aportar una cantidad útil a lo que sumen los elementos que siguen; en cambio, al ser negativa la suma de un subarreglo, no tiene sentido verlo como prefijo de un subarreglo candidato ya que su suma sólo le restaría a lo que puedan aportar los elementos siguientes. En este sentido, el poner en cero el acumulador luego de encontrar una acumulación negativa equivale a descartar el subarreglo que se estaba evaluando desde la última vez que el acumulador se anuló y empezar a evaluar subarreglos que comienzan inmediatamente después del descartado.

% Para verlo en más detalle, supongamos el caso de un arreglo $A$ para el cual el algoritmo vuelve a poner en cero su acumulador por primera vez en la iteración $i_1 > 0$. Es decir, el prefijo de longitud $i_1$ de $A$ es el primer subarreglo de suma negativa que nuestro algoritmo contempla. Si luego el algoritmo vuelve encontrar una acumulación negativa en la iteración $i_2 > i_1$, tenemos que vuelve a poner en cero su acumulador luego del prefijo de $A$ de longitud $i_1 + i_2$, formado por el prefijo de $A$ de longitud $i_1$ y los $i_2$ elementos que le siguen, cuya suma también es negativa. Si luego ocurre lo análogo para un índice $i_3 > i_2$, tendremos que el algoritmo vuelve a anular su acumulador luego del prefijo de longitud $i_1 + i_2 + i_3$ de $A$ que suma una cantidad negativa también, y así.

\subsection{Correctitud}

Sea $A$ un arreglo y $s$ el resultado correcto del problema para $A$, es decir, la máxima suma posible para subarreglos de $A$. Esta cantidad óptima existe ya que el conjunto de posibles sumas de subarreglos es un conjunto finito de números enteros, y por ende tiene un máximo. Observemos que $s$ nunca será un valor negativo, ya que en ese caso podría tomarse el subarreglo vacío y tener una suma de un subarreglo mayor a $s$, lo cual sería absurdo habiendo supuesto óptimo a $s$.

Sea además $S$ un subarreglo minimal de $A$ que realiza la suma óptima $s$ y tal que su primer elemento está en la posición $i$ de $A$. Notar que ningún prefijo de $S$ puede tener una suma negativa; de existir tal prefijo, podríamos descartarlo y obtener con lo que queda de $S$ una suma mayor a $s$, lo que sería absurdo dado que $s$ se supuso óptimo. Además, $S$ no tiene prefijos que sumen $0$ ya que se supuso minimal.

Vamos a ver la correctitud de nuestro algoritmo demostrando que el mismo necesariamente considera la suma del subarreglo $S$ como candidata. Primero, demostramos que el algoritmo comienza a iterar los elementos de $S$ teniendo \textit{acumulador} en $0$.

\begin{itemize}
    \item \textbf{Caso} $\mathbf{i = 0}$. En este caso, el subarreglo $S$ es un prefijo de $A$. Por lo tanto, nuestro algoritmo lo contempla necesariamente ya que comienza a iterar al principio de $A$, y lo hace con \textit{acumulador} en $0$ dado que se inicializa en ese valor.

    \item \textbf{Caso} $\mathbf{i > 0}$. Sea $S_{pre}$ el prefijo de $A$ cuyo último elemento es el $i-1$\textit{-ésimo} de $A$. En este caso, todo sufijo de $S_{pre}$ suma necesariamente una cantidad menor o igual a $0$, ya que si existiera uno que suma una cantidad positiva, podríamos tomar el subarreglo que resulta de concatenar ese sufijo con $S$ y obtener una suma mayor a $s$. Por lo tanto, podemos afirmar que al comenzar a recorrer el subarreglo $S$, \textit{acumulador} está en $0$.
\end{itemize}

Como $S$ no tiene prefijos que sumen una cantidad negativa o nula, una vez que el ciclo comienza a iterar sus elementos \textit{acumulador} nunca se pondrá en cero mientras el ciclo recorra $S$. Por ende, al finalizar el recorrido de $S$, \textit{mayorAlcanzado} tendrá el valor $s$. Además, dicha variable sólo cambia de valor en iteraciones en las que \textit{acumulador} tiene un valor mayor aún. Dado que \textit{acumulador} sólo toma valores de sumas de subarreglos de $A$ y siendo por hipótesis $s$ mayor o igual a todos esos valores, podemos concluir que, al finalizar la ejecución del ciclo, \textit{mayorAlcanzado} es igual a $s$. Como el valor de retorno del algoritmo es el que tiene \textit{mayorAlcanzado} luego del ciclo, el mismo devuelve $s$.

\subsection{Complejidad}

El algoritmo recorre el arreglo de entrada exactamente una vez en su enteridad, siguiendo un ciclo cuya iteración consiste en una cantidad constante de comparaciones y asignaciones. Las operaciones fuera del ciclo se limitan también a un tiempo constante. Por lo tanto, si $n$ es el número de elementos del arreglo, la complejidad temporal del algoritmo es $\Theta(n)$, que cumple con la complejidad requerida de $O(n)$.

\subsection{Casos de prueba}

Utilizamos las siguientes familias de casos de prueba. Se entrega con los fuentes del trabajo un código que testea el programa correspondiente con determinadas instancias de cada una de ellas.

\begin{itemize}
    \item El caso del arreglo vacío, en el que la suma óptima es trivialmente $0$.
    \item Casos no-vacíos en los que la suma óptima es $0$. Por ejemplo, un caso de números negativos en el que la suma óptima la tiene el subarreglo vacío, como
    \[
        \{-1,-1,-1,-1,-1\}
    \]
    o casos de elementos negativos o nulos en los que más subarreglos realizan la suma óptima, como
    \[
        \{-1,-1,0,-1,0\}
    \]
    \item Casos en los que la suma óptima es la de todo el arreglo. Por ejemplo, arreglos de todos números positivos o nulos como
    \[
        \{1,2,0,3,0,4\}
    \]
    o de números positivos y negativos como
    \[
        \{1,2,3,4,5,-1,-2,-3,6,-5,1,2,3,4\}
    \]
    \item Casos en los que la suma óptima es la de un prefijo no-vacío, como por ejemplo
    \[
        \{1,2,3,4,5,-1,-2,-3,-4,-5,1,2,3,4\}
    \]
    \item Casos en los que el subarreglo que realiza la suma óptima no es prefijo ni sufijo, como
    \[
        \{1,2,-7,1,2,3,4,5,-16,2\}
    \]
    \item Casos de gran magnitud de elementos para verificar que el algoritmo los procesa correctamente en tiempo razonable. Específicamente, se utilizaron casos de arreglos del tamaño máximo sugerido por la cátedra ($10^5$) de enteros no-negativos generados aleatoriamente. Elegimos utilizar números no-negativos como una forma de asegurarnos de conocer el resultado correcto para esos casos aleatorios, que según la teoría corresponde en todos ellos a la suma de todos sus elementos (al ser positivos o nulos, quitar cualquier combinación de ellos no puede aumentar ese valor).

\end{itemize}
